\documentclass[twocolumn]{article}
\usepackage{latex8}

\begin{document}
\title{Using Conference Styles with \LaTeX}
\author{Dr. Milaan Parmar}
\maketitle

\begin{abstract} 
This is some abstract text. This has been included for demonstration only. This is why it's being kept brief. 
\end{abstract} 


\section{Introduction}
Introduction text here. Smartphones are increasingly being used to store personal information as well as to access sensitive data from the Internet and the cloud. Establishment of the identity of a user requesting information from smartphones is a prerequisite for  secure systems in such scenarios. In the past, keystroke-based user identification has been successfully deployed on production-level mobile devices to mitigate the risks associated with naive username/password based authentication. However, these approaches have two major limitations: they are not applicable to services where authentication occurs outside the domain of the mobile
device such as web-based services; and they often overly tax the limited computational capabilities of mobile devices. In this paper, we propose a protocol for keystroke dynamics analysis which allows web-based applications to make use of remote attestation and delegated keystroke analysis. The end result is an efficient keystroke-based user identification mechanism that strengthens traditional password protected services
while mitigating the risks of user profiling by collaborating malicious web
services. We present a prototype implementation of our protocol using
the popular Android operating system for smartphones.

\section{Background} 
Smartphones are increasingly being used to store personal information as well as to access sensitive data from the Internet and the cloud. Establishment of the identity of a user requesting information from smartphones is a prerequisite for  secure systems in such scenarios. In the past, keystroke-based user identification has been successfully deployed on production-level mobile devices to mitigate the risks associated with naive username/password based authentication. However, these approaches have two major limitations: they are not applicable to services where authentication occurs outside the domain of the mobile
device such as web-based services; and they often overly tax the limited computational capabilities of mobile devices. In this paper, we propose a protocol for keystroke dynamics analysis which allows web-based applications to make use of remote attestation and\cite{nauman2011using} delegated keystroke analysis. The end result is an efficient keystroke-based user identification mechanism that strengthens traditional password protected services
while mitigating the risks of user profiling by \cite{se-cs-collab:nauman10}collaborating malicious web
services. We present a prototype implementation of our protocol using
the popular Android operating system for smartphones.\cite{seo2011user}

\subsection{Some Related Work} 
Establishment of the identity of a user requesting information from smartphones is a prerequisite for  secure systems in such scenarios. In the past, keystroke-based user identification has been successfully deployed on production-level mobile devices to mitigate the risks associated with naive username/password based authentication. However, these approaches have two major limitations: they are not applicable to services where authentication occurs outside the domain of the mobile
device such as web-based services; and they often overly tax the limited computational capabilities of mobile devices. In this paper, we propose a protocol for keystroke dynamics analysis which allows web-based applications to make use of remote attestation and delegated keystroke analysis.

\section{Conclusions}
In the past, keystroke-based user identification has been successfully deployed on production-level mobile devices to mitigate the risks associated with naive username/password based authentication. However, these approaches have two major limitations: they are not applicable to services where authentication occurs outside the domain of the mobile
device such as web-based services.


%\section{Changing Bibliography Styles}
%To change bibliography styles, you need to get a \verb|.bst| file for the style. For example, you can see the IEEE bib style file, ACM bib style file and Springer bib style file in the same directory as this file. 

% -------------------------------------------------------------------
% add bibliography-related commands here 
\bibliographystyle{latex8}
\bibliography{bibfile}   

\end{document}